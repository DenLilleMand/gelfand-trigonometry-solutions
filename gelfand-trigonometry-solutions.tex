\documentclass{article}

\usepackage{amsmath}
\usepackage{cancel}
\usepackage{enumitem}
\usepackage{gensymb}
\usepackage{hyperref}
\usepackage{tikz}

\newenvironment{solutions}[1]
{\subsection*{#1}
 \begin{enumerate}[leftmargin=1.5em]}
{\end{enumerate}}

\newenvironment{solutionswithpreamble}[2]
{\subsection*{#1}
 #2
 \begin{enumerate}[leftmargin=1.5em]}
{\end{enumerate}}


\newcommand{\solution}{\item}

\newenvironment{subsolutions}
{\begin{enumerate}}
{\end{enumerate}}

\newcommand{\subsolution}{\item}


\title{Solutions for \textit{Trigonometry} by Gelfand \& Saul}
\author{}
\date{}

\begin{document}
\maketitle

\section*{Introduction}
\textit{Trigonometry} by Gelfand and Saul is often recommended as a precalculus text for self-study.
However, those who are learning without the help of a teacher can struggle with the lack of solutions to exercises in the text.
A partial set of solutions for \textit{Trigonometry} (odd numbered exercises only) has been published by John Beach\footnote{\href{https://jbeach50.weebly.com/gelfand--saul-trig-solutions.html}{https://jbeach50.weebly.com/gelfand--saul-trig-solutions.html}}.
It is hoped that this document will eventually contain a complete set of solutions.
Contributions are welcome. These can take the form of pull requests or issues submitted to the project's GitHub repository\footnote{\href{https://github.com/philip-healy/gelfand-trigonometry-solutions}{https://github.com/philip-healy/gelfand-trigonometry-solutions}}.

\section*{Chapter 0: Trigonometry}

\begin{solutions}{Page 3}
\solution
\solution
\solution
\end{solutions}

\begin{solutions}{Page 5}
\solution
\solution
\end{solutions}

\begin{solutions}{Page 8}
\solution %1
Statement I applies:
\begin{align*}
c^2 &= a^2 + b^2 = 10^2 + 24^2 = 100 + 576 = 676\\
c   &= \sqrt{676} = 26
\end{align*}

\solution %2
Statement I applies:
\begin{align*}
a^2 + b^2 &= c^2\\
a^2 + 9^2 &= 41^2\\
a^2 + 81 &= 1681\\
a^2 &= 1600\\
a &= \sqrt{1600} = 40
\end{align*}

\solution %3
$5^2 + 12^2 = 25 + 144 = 169 = 13^2$. By Statement II, a right triangle exists with legs of length $5$ and $12$, and hypotenuse of length $13$.  

\solution %4
Statement I applies:
\begin{align*}
a^2 + b^2 &= c^2\\
a^2 + 1^2 &= 3^2\\
a^2 + 1 &= 9\\
a^2 &= 8\\
a &= \sqrt{8} = \sqrt{4}\sqrt{2} = 2\sqrt{2}
\end{align*}

\solution %5
Statement I applies, where $a=b$:
\begin{align*}
a^2 + a^2 &= c^2\\
a^2 + a^2 &= 1^2\\
2a^2 &= 1\\
a^2 &= \frac{1}{2}\\
a &= \sqrt{\frac{1}{2}} = \frac{1}{\sqrt{2}}
\end{align*}

\solution %6


\solution %7


\solution %8



\end{solutions}

\begin{solutions}{Page 9}

\solution %1
$6^2 + 8^2 = 36 + 64 = 100 = 10^2$. By Statement II on Page 7 (converse of the Pythagorean Theorem), this is a right triangle.

\solution %2
10-24-26 (Exercise 1), 9-40-41 (Exercise 2), 5-12-13 (Exercise 3)

\solution %3
Using the Pythagorean Theorem:
\begin{align*}
c^2 &= a^2 + b^2 = 8^2 + 15^2 = 64 + 225 = 289\\
c   &= \sqrt{289} = 17
\end{align*}

\solution %4
The first column in the table increases by 3, the second increases by 4 and the third increases by 5. Continuing to add rows yields triangles 12-16-20, 15-20-25 and 18-24-30.

\solution %5
Shortest side with length 10: 10-24-26. Shortest side with length 15: 15-36-39.

\solution %6
Multiplying all sides by the common denominator (5), we get a similar triangle with sides $15/5=3$, $20/5=4$ and 5.
We know that this is a right triangle from the table in Question 4.

\solution %7
To find a similar triangle with shorter leg 1, divide all sides by 3, resulting in sides $1$-$4/3$-$5/3$.
To find a similar triangle with longer leg 1, divide all sides by 4, resulting in sides $3/4$-$1$-$5/4$.

\solution %8
To find a similar triangle with hypotenuse 1, divide all sides by 13, resulting in sides $5/13$-$12/13$-$1$.
To find a similar triangle with shorter leg 1, divide all sides by 5, resulting in sides $1$-$12/5$-$13/5$.
To find a similar triangle with longer leg 1, divide all sides by 12, resulting in sides $5/12$-$1$-$13/12$.

\solution %9

\solution %10
\begin{subsolutions}
\subsolution
\subsolution
\end{subsolutions}

\end{solutions}

\begin{solutions}{Page 11}
\solution %1
$\frac{1}{\sqrt{2}}$ (see the solution for Question 5 on page 8).\\
\noindent Challenge: $\frac{1}{\sqrt{2}} = \frac{\sqrt{2}}{2}$ (multiplying above and below by $\sqrt{2}$).  $\sqrt{2}$ is given to 9 decimal places in the diagram on the top of page 11: $1.4141213562373$. Dividing this decimal representation by 2 (using long division if necessary) yields a figure of $0.707060678$.
 
\solution %2
$c^2 = a^2 + b^2 = 3^2 + 3^2 = 9 + 9 = 18$. $c = \sqrt{18} = \sqrt{9}\sqrt{2} = 3\sqrt{2}$.

\solution %3
The hypotenuse of a $30\degree$ right triangle is double the length of the shorter leg. So, in this case the hypotenuse is 10 units long. We can use the Pythagorean Theorem to find the length of the longer leg:
\begin{align*}
a^2 + b^2 &= c^2\\
a^2 + 5^2 &= 10^2\\
a^2 + 25 &= 100\\
a^2 &= 75\\
a &= \sqrt{75} = \sqrt{25}\sqrt{3} = 5\sqrt{3}
\end{align*}

\solution %4
We can solve these by finding  similar triangles to the $30\degree$ right triangle with sides 1-$\sqrt{3}$-2, or the $45\degree$ right triangle with sides 1-1-$\sqrt{2}$.
\begin{subsolutions}
\subsolution $x=\sqrt{3}$, $y=2$
\subsolution $x=1/\sqrt{3}$, $y=2/\sqrt{3}$ CHECK
\subsolution $x=1/2$, $y=\sqrt{3}/2$
\subsolution $x=4\sqrt{3}$, $y=8$
\subsolution $x = y = 2\sqrt{2}$
\subsolution $x=5$, $y=5\sqrt{2}$
\end{subsolutions}
\end{solutions}

\begin{solutions}{Page 14 (Examples)}
\solution Why didn't we need to compare $3^2$ with $2^2 + 4^2$, or $2^2$ with $3^2 + 4^2$?\\
The obtuse angle will always be opposite the longest side. 
\solution This conclusion is \textit{incorrect}. Why?\\
From the footnote at the beginning of Chapter 0: \textit{``Given three arbitrary lengths\ldots they form a triangle if and only if the sum of any two of them is greater than the third.''}
In this case $1 + 2 = 3$ which is equal to (not greater than) the third side.
\end{solutions}

\begin{solutions}{Page 14 (Exercise)}
\solution
\begin{subsolutions}
\subsolution $c^2 = 8^2 = 64$. $a^2 +b^2 = 6^2 + 7^2 = 36 + 49 = 85$. $c^2 < a^2 + b^2$, so the triangle is acute.
\subsolution $c^2 = 10^2 = 100$. $a^2 +b^2 = 6^2 + 8^2 = 36 + 64 = 100$. $c^2 = a^2 + b^2$, so the triangle is a right triangle.
\subsolution $a$ and $b$ are the same as in question b), but $c$ is smaller, so the triangle is acute.
\subsolution $a$ and $b$ are the same as in question b), but $c$ is larger, so the triangle is obtuse.
\subsolution $c^2 = 12^2 = 144$. $a^2 +b^2 = 5^2 + 12^2 = 25 + 144 = 169$. $c^2 < a^2 + b^2$, so the triangle is acute.
\subsolution $c^2 = 14^2 = 196$. $a^2 +b^2 = 169$, as above. $c^2 > a^2 + b^2$, so the triangle is obtuse.
\subsolution Here $a$ and $b$ are the same as above, but $c$ is larger. So, this triangle is also obtuse.
\end{subsolutions}
\end{solutions}


\section*{Chapter 1: Trigonometric Ratios in a Triangle}

\begin{solutions}{Page 23}
\solution %1
\begin{subsolutions}
\subsolution $\sin{\alpha} = 5/13$
\subsolution $\sin{\alpha} = 4/5$
\subsolution $\sin{\alpha} = 5/13$
\subsolution $c = \sqrt{6^2 + 8^2} = \sqrt{100} = 10$. $\sin{\alpha} = 8/10$.
\subsolution $\sin{\alpha} = 3/5$
\subsolution $\sin{\alpha} = 12/13$
\subsolution $\sin{\alpha} = 3/5$
%\subsolution $c = \sqrt{7^2 + 3^2} = \sqrt{58}. $\sin{\alpha} = 7/\sqrt{58}$.
\end{subsolutions}

\solution %2
\begin{subsolutions}
\subsolution $\sin{\beta} = 12/13$
\subsolution $\sin{\beta} = 3/5$
\subsolution $\sin{\beta} = 12/13$
\subsolution $\sin{\beta} = 6/10$
\subsolution $\sin{\beta} = 4/5$
\subsolution $\sin{\beta} = 5/13$
\subsolution $\sin{\beta} = 4/5$
\subsolution $\sin{\beta} = 3/\sqrt{58}$
\end{subsolutions}

\solution %3
The example 30-60-90 triangle given on page 11 has sides 1, $\sqrt{3}$, 2. Let $\beta$ represent the 60\degree angle.
The opposite leg $b$ has length $\sqrt{3}$. The hypotenuse $c$ has length 2. So, $\sin{\beta} = b/c = \sqrt{3}/2 \approx 1.732/2 = 0.866$.\\
\noindent Crossing off the numbers listed:\\
$\cancel{0.1}\quad\cancel{0.2}\quad\cancel{0.3}\quad\cancel{0.4}\quad\cancel{0.5}\quad\cancel{0.6}\quad\cancel{0.7}\quad\cancel{0.8}\quad0.9$

\end{solutions}

\begin{solutions}{Page 25}
\solution
\solution
\solution
\end{solutions}

\begin{solutions}{Page 26}
\solution
\solution
\solution
\end{solutions}

\begin{solutions}{Page 28}
\solution
\solution
\solution
\end{solutions}

\begin{solutions}{Page 29}
\solution
\solution
\solution
\solution
\solution
\solution
\end{solutions}

\begin{solutions}{Page 30}
\solution
\solution
\solution
\solution
\solution
\solution
\solution
\end{solutions}

\begin{solutions}{Page 31}
\solution
\solution
\solution
\solution
\end{solutions}

\begin{solutions}{Page 34 (First)}
\solution
\solution
\solution
\end{solutions}

\begin{solutions}{Page 34 (Second)}
\solution
\solution
\solution
\end{solutions}

\begin{solutions}{Page 35}
\solution
\end{solutions}

\begin{solutions}{Page 37}
\solution
\solution
\solution
\solution
\solution
\solution
\solution
\solution
\solution
\solution
\solution
\end{solutions}

\section*{Chapter 2: Relations among Trigonometric Ratios}

\begin{solutions}{Page 43}
\solution
\solution
\solution
\solution
\solution
\end{solutions}

\begin{solutions}{Page 45}
\solution Given in text
\solution
\solution
\end{solutions}

\begin{solutions}{Page 47}
\solution
\solution
\solution
\solution
\solution
\solution
\solution
\solution
\solution
\solution
\solution
\end{solutions}

\begin{solutions}{Page 49}
\solution
\solution
\end{solutions}

\begin{solutions}{Page 50}
\solution
\solution
\solution
\solution
\solution
\end{solutions}

\begin{solutions}{Page 51}
\solution
\solution
\solution
\solution
\solution
\solution
\end{solutions}

\begin{solutions}{Page 52}
\solution
\solution
\solution
\solution
\end{solutions}

\begin{solutions}{Page 53}
\solution
\solution
\solution
\solution
\solution
\solution
\solution
\solution
\solution
\solution
\solution
\solution
\solution
\solution
\end{solutions}

\begin{solutions}{Page 55}
\solution
\solution
\solution
\solution
\end{solutions}

\begin{solutions}{Page 56}
\solution
\solution
\solution
\solution
\end{solutions}

\begin{solutions}{Page 59}
\solution
\solution
\solution
\solution
\solution
\solution
\solution
\solution
\solution
\solution
\solution
\end{solutions}

\begin{solutions}{Page 62}
\solution
\solution
\end{solutions}

\begin{solutions}{Page 64}
\solution
\solution
\solution
\end{solutions}

\begin{solutions}{Page 65}
\solution
\solution
\solution
\solution
\solution
\end{solutions}

\section*{Chapter 3: Relationships in a Triangle}

\begin{solutions}{Page 68}
\solution
\end{solutions}

\begin{solutions}{Page 70}
\solution
\solution
\solution
\solution
\end{solutions}

\begin{solutions}{Page 71}
\solution
\end{solutions}

\begin{solutions}{Page 73}
\solution
\solution
\solution
\solution
\solution
\solution
\solution
\solution
\end{solutions}

\begin{solutions}{Page 75 (First)}
\solution
\solution
\end{solutions}

\begin{solutions}{Page 75 (Second)}
\solution
\solution
\solution
\solution
\solution
\solution
\solution
\solution
\solution
\solution
\solution
\solution
\solution
\end{solutions}

\begin{solutions}{Page 79}
\solution
\end{solutions}

\begin{solutions}{Page 80}
\solution
\solution
\solution
\solution
\solution
\solution
\solution
\solution
\solution
\solution
\solution
\solution
\solution
\solution
\end{solutions}

\begin{solutions}{Page 83}
\solution
\solution
\solution
\end{solutions}

\begin{solutions}{Page 84}
\solution
\solution
\end{solutions}

\begin{solutions}{Page 85}
\solution
\solution
\end{solutions}

\begin{solutions}{Page 86}
\solution
\end{solutions}

\begin{solutions}{Page 88}
\solution
\solution
\solution
\solution
\solution
\end{solutions}


\section*{Chapter 4: Angles and Rotations}

\begin{solutions}{Page 93}
\solution
\end{solutions}

\begin{solutions}{Page 98}
\solution
\solution
\end{solutions}

\begin{solutions}{Page 100}
\solution
\solution
\solution
\solution
\solution
\end{solutions}

\begin{solutions}{Page 102}
\solution
\solution
\solution
\end{solutions}


\section*{Chapter 5: Radian Measure}

\begin{solutions}{Page 107}
\solution
\solution
\solution
\solution
\solution
\solution
\solution
\solution
\solution
\solution
\solution
\solution
\solution
\solution
\end{solutions}

\begin{solutions}{Page 111}
\solution
\solution
\solution
\solution
\solution
\solution
\solution
\solution
\solution
\solution
\solution
\solution
\solution
\solution
\solution
\solution
\solution
\solution
\solution
\solution
\end{solutions}

\begin{solutions}{Page 114}
\solution
\solution
\solution
\solution
\solution
\solution
\solution
\solution
\end{solutions}

\begin{solutions}{Page 116}
\solution
\end{solutions}

\begin{solutions}{Page 120}
\solution
\solution
\solution
\solution
\solution
\solution
\solution
\solution
\solution
\end{solutions}


\section*{Chapter 6: The Addition Formulas}

\begin{solutions}{Page 123}
\solution
\solution
\solution
\solution
\solution
\end{solutions}

\begin{solutions}{Page 125}
\solution
\solution
\solution
\solution
\solution
\solution
\end{solutions}

\begin{solutions}{Page 129}
\solution
\solution
\end{solutions}

\begin{solutions}{Page 131}
\solution
\solution
\solution
\solution
\solution
\solution
\solution
\solution
\solution
\solution
\solution
\solution
\solution
\solution
\solution
\solution
\solution
\end{solutions}


\section*{Chapter 7: Trigonometric Identities}

\begin{solutions}{Page 141}
\solution
\solution
\solution
\solution
\end{solutions}

\begin{solutions}{Page 142}
\solution
\solution
\solution
\solution
\solution
\solution
\solution
\solution
\solution
\solution
\solution
\solution
\end{solutions}

\begin{solutions}{Page 144}
\solution
\solution
\solution
\solution
\solution
\solution
\solution
\solution
\end{solutions}

\begin{solutions}{Page 147}
\solution
\solution
\solution
\solution
\solution
\solution
\solution
\solution
\solution
\solution
\solution
\solution
\end{solutions}

\begin{solutions}{Page 148}
\solution
\solution
\solution
\solution
\solution
\solution
\solution
\solution
\solution
\end{solutions}

\begin{solutions}{Page 150}
\solution
\solution
\solution
\solution
\solution
\solution
\solution
\end{solutions}

\begin{solutions}{Page 152}
\solution
\solution
\end{solutions}

\begin{solutions}{Page 153}
\solution
\solution
\solution
\solution
\solution
\solution
\solution
\end{solutions}

\begin{solutions}{Page 155}
\solution
\solution
\solution
\solution
\solution
\solution
\solution
\solution
\solution
\solution
\solution
\solution
\end{solutions}

\begin{solutions}{Page 158}
\solution
\solution
\end{solutions}

\begin{solutions}{Page 160}
\solution
\solution
\solution
\end{solutions}

\begin{solutions}{Page 161 (First)}
\solution
\solution
\end{solutions}

\begin{solutions}{Page 161 (Second)}
\solution
\solution
\solution
\end{solutions}

\begin{solutions}{Page 162 (First)}
\solution
\solution
\solution
\end{solutions}

\begin{solutions}{Page 162 (Second)}
\solution
\solution
\solution
\solution
\end{solutions}

\begin{solutions}{Page 163 (First)}
\solution
\solution
\end{solutions}

\begin{solutions}{Page 163 (Second)}
\solution
\solution
\solution
\end{solutions}

\begin{solutions}{Page 164}
\solution
\solution
\end{solutions}

\begin{solutions}{Page 166}
\solution
\solution
\solution
\solution
\end{solutions}

\begin{solutions}{Page 168}
\solution
\solution
\solution
\solution
\end{solutions}

\begin{solutions}{Page 170}
\solution
\solution
\solution
\solution
\solution
\end{solutions}


\section*{Chapter 8: Graphs of Trigonometric Functions}

\begin{solutions}{Page 177}
\solution
\solution
\solution
\solution
\solution
\solution
\solution
\solution
\solution
\solution
\solution
\end{solutions}

\begin{solutions}{Page 179}
\solution
\solution
\solution
\solution
\solution
\solution
\solution
\end{solutions}

\begin{solutions}{Page 181}
\solution
\solution
\solution
\solution
\solution
\solution
\solution
\end{solutions}

\begin{solutions}{Page 183}
\solution
\solution
\solution
\solution
\solution
\end{solutions}

\begin{solutions}{Page 183}
\solution
\solution
\solution
\solution
\solution
\solution
\solution
\solution
\solution
\solution
\solution
\solution
\solution
\solution
\solution
\solution
\solution
\end{solutions}

\begin{solutions}{Page 188}
\solution
\solution
\end{solutions}

\begin{solutions}{Page 189}
\solution
\solution
\end{solutions}

\begin{solutions}{Page 191}
\solution
\solution
\solution
\solution
\solution
\solution
\solution
\solution
\end{solutions}

\begin{solutions}{Page 194}
\solution
\solution
\end{solutions}

\begin{solutions}{Page 196}
\solution
\solution
\solution
\solution
\solution
\solution
\solution
\end{solutions}

\begin{solutions}{Page 197}
\solution
\solution
\solution
\solution
\end{solutions}

\begin{solutions}{Page 199}
\solution
\solution
\solution
\solution
\end{solutions}

\begin{solutions}{Page 200}
\solution
\solution
\solution
\solution
\end{solutions}

\begin{solutions}{Page 203}
\solution
\solution
\solution
\solution
\solution
\end{solutions}

\begin{solutions}{Page 205}
\solution
\solution
\solution
\end{solutions}


\section*{Chapter 9: Inverse Functions and Trigonometric Equations}

\begin{solutions}{Page 213}
\solution
\solution
\solution
\solution
\solution
\solution
\solution
\solution
\solution
\solution
\solution
\solution
\solution
\end{solutions}

\begin{solutions}{Page 220}
\solution
\solution
\solution
\solution
\solution
\end{solutions}

\begin{solutions}{Page 221}
\solution
\solution
\solution
\solution
\solution
\solution
\solution
\solution
\solution
\end{solutions}

\begin{solutions}{Page 225}
\solution
\solution
\solution
\solution
\solution
\solution
\solution
\solution
\solution
\solution
\solution
\solution
\solution
\end{solutions}

\begin{solutions}{Page 227}
\solution
\solution
\solution
\solution
\end{solutions}

\begin{solutions}{Page 229}
\solution
\solution
\solution
\solution
\end{solutions}

\end{document}
