\documentclass{article}

\usepackage{amsmath}
\usepackage{cancel}
\usepackage{catchfile}
\usepackage{enumitem}
\usepackage{gensymb}
\usepackage{hyperref}
\usepackage{makecell}
\usepackage{tabularx}
\usepackage{tikz}
\usepackage{verbatim}
\usepackage{xstring}

% Add license
\usepackage{ccicons}
\usepackage[
    type={CC},
    modifier={by-sa},
    version={4.0},
]{doclicense}

% Configure hyperref
\hypersetup{
    colorlinks=true,
    linkcolor=blue,
    urlcolor=blue
}


% Solutions formatting
\newenvironment{solutions}[1]
{\subsection*{#1}
 \begin{enumerate}[leftmargin=1.5em]}
{\end{enumerate}}

\newenvironment{solutionswithpreamble}[2]
{\subsection*{#1}
 #2
 \begin{enumerate}[leftmargin=1.5em]}
{\end{enumerate}}


\newcommand{\solution}{\item}

\newenvironment{subsolutions}
{\begin{enumerate}}
{\end{enumerate}}

\newcommand{\subsolution}{\item}


% Git info
% See https://tex.stackexchange.com/questions/455396/how-to-include-the-current-git-commit-id-and-branch-in-my-document
\CatchFileDef{\headfull}{.git/HEAD}{}
\StrGobbleRight{\headfull}{1}[\head]
\StrBehind[2]{\head}{/}[\branch]
\CatchFileDef{\commit}{.git/refs/heads/\branch}{}
\StrLeft{\commit}{7}[\abbrevcommit]

\title{Solutions for \textit{Trigonometry} by Gelfand \& Saul}
\author{}
\date{Git commit \texttt{\abbrevcommit}. Published \today.}

\begin{document}
\maketitle
\doclicenseThis

\section*{Introduction}
\textit{Trigonometry} by Gelfand and Saul is often recommended as a precalculus text for self-study.
However, those who are learning without the help of a teacher can struggle with the lack of solutions to exercises in the text.
A partial set of solutions for \textit{Trigonometry} (odd numbered exercises only) has been published by John Beach\footnote{\href{https://jbeach50.weebly.com/gelfand--saul-trig-solutions.html}{https://jbeach50.weebly.com/gelfand--saul-trig-solutions.html}}.
It is hoped that this document will eventually contain a complete set of solutions.
Contributions are welcome. These can take the form of pull requests or issues submitted to the project's GitHub repository\footnote{\href{https://github.com/philip-healy/gelfand-trigonometry-solutions}{https://github.com/philip-healy/gelfand-trigonometry-solutions}}.

\section*{Chapter 0: Trigonometry}

\begin{comment}
\begin{solutions}{Page 3}
\solution
\solution
\solution
\end{solutions}

\begin{solutions}{Page 5}
\solution
\solution
\end{solutions}
\end{comment}

\begin{solutions}{Page 8}
\solution %1
Statement I applies:
\begin{align*}
c^2 &= a^2 + b^2 = 10^2 + 24^2 = 100 + 576 = 676\\
c   &= \sqrt{676} = 26
\end{align*}

\solution %2
Statement I applies:
\begin{align*}
a^2 + b^2 &= c^2\\
a^2 + 9^2 &= 41^2\\
a^2 + 81 &= 1681\\
a^2 &= 1600\\
a &= \sqrt{1600} = 40
\end{align*}

\solution %3
$5^2 + 12^2 = 25 + 144 = 169 = 13^2$. By Statement II, a right triangle exists with legs of length $5$ and $12$, and hypotenuse of length $13$.  

\solution %4
Statement I applies:
\begin{align*}
a^2 + b^2 &= c^2\\
a^2 + 1^2 &= 3^2\\
a^2 + 1 &= 9\\
a^2 &= 8\\
a &= \sqrt{8} = \sqrt{4}\sqrt{2} = 2\sqrt{2}
\end{align*}

\solution %5
Statement I applies, where $a=b$:
\begin{align*}
a^2 + a^2 &= c^2\\
a^2 + a^2 &= 1^2\\
2a^2 &= 1\\
a^2 &= \frac{1}{2}\\
a &= \sqrt{\frac{1}{2}} = \frac{\sqrt{1}}{\sqrt{2}} = \frac{1}{\sqrt{2}}
\end{align*}

\solution %6
From the diagram at the bottom of Page 11, we can see the shorter leg is half the length of the hypotenuse.
So in this instance the shorter leg has length $1/2$. We can use Statement 1 to find the length of the longer leg:
\begin{align*}
a^2 + b^2 &= c^2\\
a^2 + \left(\frac{1}{2}\right)^2 &= 1^2\\
a^2 + \frac{1}{4} &= 1\\
a^2 &= \frac{3}{4}\\
a &= \sqrt{\frac{3}{4}} = \frac{\sqrt{3}}{\sqrt{4}} = \frac{\sqrt{3}}{2}
\end{align*}

\solution %7
For any point $Y$, we can draw a triangle with sides $AY$, $BY$ and $AB$.
Let $a$ be the length of side $AY$, $b$ be the length of side $BY$ and $c$ be the length of side $AB$.
According to Statement II, the subset of these triangles where $a^2 + b^2 = c^2$ are right triangles with legs of length $a$ and $b$ and hypotenuse $c$.
Let $X$ be the subset of $Y$ that are vertices of these right triangles.
This set of points describes a circle with its centre at the midpoint of $AB$, and radius $AB/2$.

\solution %8

\end{solutions}


\begin{solutions}{Page 9}
\solution %1
$6^2 + 8^2 = 36 + 64 = 100 = 10^2$. By Statement II on Page 7 (converse of the Pythagorean Theorem), this is a right triangle.

\solution %2
10-24-26 (Exercise 1), 9-40-41 (Exercise 2), 5-12-13 (Exercise 3)

\solution %3
Using the Pythagorean Theorem:
\begin{align*}
c^2 &= a^2 + b^2 = 8^2 + 15^2 = 64 + 225 = 289\\
c   &= \sqrt{289} = 17
\end{align*}

\solution %4
The first column in the table increases by 3, the second increases by 4 and the third increases by 5. Continuing to add rows yields triangles 12-16-20, 15-20-25 and 18-24-30.

\solution %5
Shortest side with length 10: 10-24-26. Shortest side with length 15: 15-36-39.

\solution %6
Multiplying all sides by the common denominator (5), we get a similar triangle with sides $15/5=3$, $20/5=4$ and 5.
We know that this is a right triangle from the table in Question 4.

\solution %7
To find a similar triangle with shorter leg 1, divide all sides by 3, resulting in sides $1$-$4/3$-$5/3$.
To find a similar triangle with longer leg 1, divide all sides by 4, resulting in sides $3/4$-$1$-$5/4$.

\solution %8
To find a similar triangle with hypotenuse 1, divide all sides by 13, resulting in sides $5/13$-$12/13$-$1$.
To find a similar triangle with shorter leg 1, divide all sides by 5, resulting in sides $1$-$12/5$-$13/5$.
To find a similar triangle with longer leg 1, divide all sides by 12, resulting in sides $5/12$-$1$-$13/12$.

\solution %9
To formula for the area of a triangle is $\frac{1}{2}bh$ where $b$ is the length of the base and $h$ is the height.
For right triangles, finding the area is easy: one leg is the base and the other leg is the height.
For other triangles, finding the height is more difficult: we need to find the length of the altitude drawn from the base.
The triangles with sides 5-12-13 and 9-12-15 are both right triangles: see Exercise 3 on Page 8 and Exercise 4 on Page 9.
The triangle with sides 13-14-15 is not a right triangle.
We can confirm this using Statement I: $a^2 + b^2 = 13^2 + 14^2 = 365$, $c^2 = 15^2 = 225, a^2 + b^2 \neq c^2$.
However, if we join the 5-12-13 and 9-12-15 triangles using their equal legs, the resulting triangle has the dimensions we are looking for: 13-14-15.
The base of this combined triangle has length $5 + 9 = 14$. We also know the length of the altitude from the base of the combined triangle: 12.
So, the area of the 13-14-15 triangle is $\frac{1}{2}\cdot14\cdot12 = 84$ units squared.

\solution %10
\begin{subsolutions}
\subsolution
%$a^2 + b^2 = 25^2 + 39^2 =  625 + 1521 = 2146$. $c^2 = 56^2 = 3136. a^2 + b^2 \neq c^2$
\subsolution
%$a^2 + b^2 = 25^2 + 39^2 =  625 + 1521 = 2146$. $c^2 = 16^2 = 256. a^2 + b^2 \neq c^2$

%16 50 56
%16^2 + 50^2 = 256 + 2500 = 2756
%56^2 = 3136
%56 + 16 = 72. 72^2 = 5184

\end{subsolutions}

\end{solutions}

\begin{solutions}{Page 11}
\solution %1
$\frac{1}{\sqrt{2}}$ (see the solution for Question 5 on page 8).\\
\noindent Challenge: $\frac{1}{\sqrt{2}} = \frac{\sqrt{2}}{2}$ (multiplying above and below by $\sqrt{2}$).  $\sqrt{2}$ is given to 9 decimal places in the diagram on the top of page 11: $1.4141213562373$. Dividing this decimal representation by 2 (using long division if necessary) yields a figure of $0.707060678$.
 
\solution %2
$c^2 = a^2 + b^2 = 3^2 + 3^2 = 9 + 9 = 18$. $c = \sqrt{18} = \sqrt{9}\sqrt{2} = 3\sqrt{2}$.

\solution %3
The hypotenuse of a $30\degree$ right triangle is double the length of the shorter leg. In this instance the hypotenuse is 10 units long. We can use the Pythagorean Theorem to find the length of the longer leg:
\begin{align*}
a^2 + b^2 &= c^2\\
a^2 + 5^2 &= 10^2\\
a^2 + 25 &= 100\\
a^2 &= 75\\
a &= \sqrt{75} = \sqrt{25}\sqrt{3} = 5\sqrt{3}
\end{align*}

\solution %4
We can solve these by finding  similar triangles to the $30\degree$ right triangle with sides 1-$\sqrt{3}$-2, or the $45\degree$ right triangle with sides 1-1-$\sqrt{2}$.
\begin{subsolutions}
\subsolution $x=\sqrt{3}$, $y=2$
\subsolution $x=\sqrt{3}$, $y=2\sqrt{3}$
\subsolution $x=1/2$, $y=\sqrt{3}/2$
\subsolution $x=4\sqrt{3}$, $y=8$
\subsolution $x = y = 2\sqrt{2}$
\subsolution $x=5$, $y=5\sqrt{2}$
\end{subsolutions}
\end{solutions}

\begin{solutions}{Page 14 (Examples)}
\solution Why didn't we need to compare $3^2$ with $2^2 + 4^2$, or $2^2$ with $3^2 + 4^2$?\\
The obtuse angle will always be opposite the longest side. 
\solution This conclusion is \textit{incorrect}. Why?\\
From the footnote at the beginning of Chapter 0: \textit{``Given three arbitrary lengths\ldots they form a triangle if and only if the sum of any two of them is greater than the third.''}
In this case $1 + 2 = 3$ which is equal to (not greater than) the third side.
\end{solutions}

\begin{solutions}{Page 14 (Exercise)}
\solution
\begin{subsolutions}
\subsolution $c^2 = 8^2 = 64$. $a^2 +b^2 = 6^2 + 7^2 = 36 + 49 = 85$. $c^2 < a^2 + b^2$, so the triangle is acute.
\subsolution $c^2 = 10^2 = 100$. $a^2 +b^2 = 6^2 + 8^2 = 36 + 64 = 100$. $c^2 = a^2 + b^2$, so the triangle is a right triangle.
\subsolution $a$ and $b$ are the same as in question b), but $c$ is smaller, so the triangle is acute.
\subsolution $a$ and $b$ are the same as in question b), but $c$ is larger, so the triangle is obtuse.
\subsolution $c^2 = 12^2 = 144$. $a^2 +b^2 = 5^2 + 12^2 = 25 + 144 = 169$. $c^2 < a^2 + b^2$, so the triangle is acute.
\subsolution $c^2 = 14^2 = 196$. $a^2 +b^2 = 169$, as above. $c^2 > a^2 + b^2$, so the triangle is obtuse.
\subsolution Here $a$ and $b$ are the same as above, but $c$ is larger. So, this triangle is also obtuse.
\end{subsolutions}
\end{solutions}


\section*{Chapter 1: Trigonometric Ratios in a Triangle}

\begin{solutions}{Page 23}
\solution %1
\begin{subsolutions}
\subsolution $\sin{\alpha} = 5/13$
\subsolution $\sin{\alpha} = 4/5$
\subsolution $\sin{\alpha} = 5/13$
\subsolution $c = \sqrt{6^2 + 8^2} = \sqrt{100} = 10$. $\sin{\alpha} = 8/10$.
\subsolution $\sin{\alpha} = 3/5$
\subsolution $\sin{\alpha} = 12/13$
\subsolution $\sin{\alpha} = 3/5$
\subsolution $c = \sqrt{7^2 + 3^2} = \sqrt{58}$. $\sin{\alpha} = 7/\sqrt{58}$.
\end{subsolutions}

\solution %2
\begin{subsolutions}
\subsolution $\sin{\beta} = 12/13$
\subsolution $\sin{\beta} = 3/5$
\subsolution $\sin{\beta} = 12/13$
\subsolution $\sin{\beta} = 6/10$
\subsolution $\sin{\beta} = 4/5$
\subsolution $\sin{\beta} = 5/13$
\subsolution $\sin{\beta} = 4/5$
\subsolution $\sin{\beta} = 3/\sqrt{58}$
\end{subsolutions}

\solution %3
The example 30-60-90 triangle given on page 11 has sides 1, $\sqrt{3}$, 2. Let $\beta$ represent the 60\degree angle.
The opposite leg $b$ has length $\sqrt{3}$. The hypotenuse $c$ has length 2. So, $\sin{\beta} = b/c = \sqrt{3}/2 \approx 1.732/2 = 0.866$.\\
\noindent Crossing off the numbers listed:\\
$\cancel{0.1}\quad\cancel{0.2}\quad\cancel{0.3}\quad\cancel{0.4}\quad\cancel{0.5}\quad\cancel{0.6}\quad\cancel{0.7}\quad\cancel{0.8}\quad0.9$

\end{solutions}

\begin{solutions}{Page 25}
\solution %1
The Altitude-on-Hypotenuse Theorem tells us that when an altitude is drawn to the hypotenuse of a right triangle, the two triangles formed are similar to the given triangle and to each other. Therefore, the triangles with sides $a$-$b$-$c$, $a$-$p$-$d$ and $d$-$b$-$q$ are similar, and the ratio for $\sin{\alpha}$ appears in all of them:
\begin{subsolutions}
\subsolution $b/c$
\subsolution $d/a$
\subsolution $q/b$
\end{subsolutions}

\solution %2
\begin{subsolutions}
\subsolution $\sin{\alpha} = h/b$
\subsolution Multiplying both sides of formula above by $b$: $h = b\sin{\alpha}$
\subsolution Substituting $b\sin{\alpha}$ for $h$, the formula for the area of $ABC$ can be rewritten as: $bc\sin{\alpha}/2$.
\subsolution $\sin{\beta} = h/a$. Rewriting this in terms of $h$: $h = a\sin{\beta}$. Substituting this for $h$ in the area formula: $ac\sin{\beta}/2$.
\subsolution Let $h_{2}$ represent the altitude from $A$ to $BC$. $sin{\beta} = h_{2}/c$. Rewriting in terms of $h_{2}$, we get $h_{2} = c\sin{\beta}$. 
\end{subsolutions}

\solution %3
\begin{subsolutions}
\subsolution
Expressing $h$ in terms of $\sin{\alpha}$ and $b$: 
\begin{align*}
\sin{\alpha} &= \frac{h}{b}\\
h &= b\sin{\alpha}
\end{align*}
Expressing $h$ in terms of $\sin{\beta}$ and $a$:
\begin{align*}
\sin{\beta} &= \frac{h}{a}\\
h &= a\sin{\beta}
\end{align*}

\subsolution
Both expresssions are equal to $h$:
\begin{equation*}
a\sin{\beta} = h = b\sin{\alpha}
\end{equation*}

\subsolution
Expressing $h_{2}$ in terms of $\sin{\beta}$ and $c$: 
\begin{align*}
\sin{\beta} &= \frac{h_{2}}{c}\\
h_{2} &= c\sin{\beta}
\end{align*}
Expressing $h_{2}$ in terms of $\sin{\gamma}$ and $b$:
\begin{align*}
\sin{\gamma} &= \frac{h_{2}}{b}\\
h_{2} &= b\sin{\gamma}
\end{align*}
Both expressions are equal to $h_{2}$:
\begin{equation*}
b\sin{\alpha} = h_{2} = c\sin{\gamma}
\end{equation*}

\subsolution
\begin{enumerate}
\item We can rewrite the result from part (b) so that the expressions on each side are fractions with sine denominators:
\begin{align*}
a\sin{\beta} &= b\sin{\alpha}\\
\frac{a\sin{\beta}}{\sin{\alpha}\sin{\beta}} &= \frac{b\sin{\alpha}}{\sin{\alpha}\sin{\beta}}\\
\frac{a}{\sin{\alpha}} &= \frac{b}{\sin{\beta}}
\end{align*}
\item We can rewrite the result from part (c) similarly:
\begin{align*}
c\sin{\beta} &= b\sin{\gamma}\\
\frac{c\sin{\beta}}{\sin{\beta}\sin{\gamma}} &= \frac{b\sin{\gamma}}{\sin{\beta}\sin{\gamma}}\\
\frac{c}{\sin{\gamma}} &= \frac{b}{\sin{\beta}}
\end{align*}
\end{enumerate}
We can derive the Law of Sines by combining results i.~and ii.~using the common expression $b/\sin{\beta}$:\\
\begin{equation*}
\frac{a}{\sin{\alpha}} = \frac{b}{\sin{\beta}} = \frac{c}{\sin{\gamma}}
\end{equation*}
\end{subsolutions}
\end{solutions}

\begin{solutions}{Page 26}
\solution %1
\begin{subsolutions}
\subsolution
$\cos{\alpha} = 12/13$.
$\cos{\beta} = 5/13$.
\subsolution
$\cos{\alpha} = 3/5$.
$\cos{\beta} = 4/5$.
\subsolution
$\cos{\alpha} = 12/13$.
$\cos{\beta} = 5/13$.
\subsolution
$\cos{\alpha} = 6/10$.
$\cos{\beta} = 8/10$.
\subsolution
$\cos{\alpha} = 4/5$.
$\cos{\beta} = 3/5$.
\subsolution
$\cos{\alpha} = 5/13$.
$\cos{\beta} = 12/13$.
\subsolution
$\cos{\alpha} = 4/5$.
$\cos{\beta} = 3/5$.
\subsolution
$\cos{\alpha} = 3/\sqrt{58}$.
$\cos{\beta} = 7/\sqrt{58}$.
\end{subsolutions}

\solution %2
\begin{subsolutions}
\subsolution
$c = \sqrt{8^2 + 6^2} = \sqrt{64 + 36} = \sqrt{100} = 10$.
$\cos{\alpha} = 8/10$.
$\cos{\beta} = 6/10$.
\subsolution
$c = \sqrt{5^2 + 12^2} = \sqrt{25 + 155} = \sqrt{169} = 13$.
$\cos{\alpha} = 12/13$.
$\cos{\beta} = 5/13$.
\subsolution
Scaling up the $1$-$\sqrt{3}$-$2$ $30\degree$ triangle gives us a value of 20 units for the length of $c$.
Next, we will use the Pythagorean Theorem to find the length of the longer leg:
\begin{align*}
a^2 + b^2 &= c^2\\
10^2 + b^2 &= 20^2\\
b^2 &= 400 - 100 = 300\\
b &= \sqrt{300} = \sqrt{100}\sqrt{3} = 10\sqrt{3}
\end{align*}
We can now find $\cos{\alpha}$ and $\cos{\beta}$:
\begin{align*}
\cos{\alpha} &= \frac{10\sqrt{3}}{20} = \frac{\sqrt{3}}{2}\\
\cos{\beta} &= \frac{10}{20} = \frac{1}{2}
\end{align*}
\subsolution
The triangle is congruent to the one above, so the solution is the same.
\subsolution
Consider the $45\degree$ right triangle with legs of length 1 and hypotenuse $\sqrt{2}$. $\cos{\alpha} = \cos{\beta} = 1/\sqrt{2}$.
\subsolution
$c = \sqrt{3^2 + 4^2} = \sqrt{9 + 16} = \sqrt{25} = 5$.
$\cos{\alpha} = 3/5$. $\cos{\beta} = 4/5$.
\subsolution
$b = x\sqrt{3}$.
$\cos{\alpha} = x\sqrt{3}/2x = \sqrt{3}/2$.
$\cos{\beta} = x/2x = 1/2$.
\end{subsolutions}

\solution %3
The Altitude-on-Hypotenuse Theorem tells us that when an altitude is drawn to the hypotenuse of a right triangle, the two triangles formed are similar to the given triangle and to each other. Therefore, the triangles with sides $a$-$b$-$c$, $a$-$p$-$d$ and $d$-$b$-$q$ are similar, and the ratio for $\cos{\alpha}$ appears in all of them:
\begin{subsolutions}
\subsolution $a/c$
\subsolution $p/a$
\subsolution $d/b$
\end{subsolutions}

\end{solutions}

\begin{solutions}{Page 28}
\solution %1
In this instance, $\alpha=29\degree$, $\beta=61\degree$, and $\alpha + \beta = 90\degree$. According to the theorem above, if $\alpha + \beta = 90\degree$, then $\sin{\alpha} = \cos{\beta}$.

\solution %2
$x = 90 - 35 = 55\degree$

\solution %3
If $\alpha + \beta = 90\degree$, then $\beta =  90\degree - \alpha$.
According to the theorem above, $\sin{\alpha} = \cos{\beta}$.
Substituting $(90- \alpha)$ for $\beta$: $\sin{\alpha} = \cos{(90 - \alpha)}$.
\end{solutions}

\begin{solutionswithpreamble}{Page 29}
{First, we need to find the length of the hypotenuse: $c = \sqrt{3^2 + 4^2} = \sqrt{9 + 16} = \sqrt{25} = 5.$}
\solution $\sin^{2}{\alpha} = \left(\frac{4}{5}\right)^2 = \frac{16}{25}$
\solution $\sin^{2}{\beta} = \left(\frac{3}{5}\right)^2 = \frac{9}{25}$
\solution $\cos^{2}{\alpha} = \left(\frac{3}{5}\right)^2 = \frac{9}{25}$ (same as $\sin^{2}{\beta}$)
\solution $\cos^{2}{\beta} = \left(\frac{4}{5}\right)^2 = \frac{16}{25}$ (same as $\sin^{2}{\alpha}$)
\solution $\sin^{2}{\alpha} + \cos^{2}{\alpha} = \frac{16}{25} + \frac{9}{25} = \frac{25}{25} = 1$
\solution $\sin^{2}{\alpha} + \cos^{2}{\beta} = \frac{16}{25} + \frac{16}{25} = \frac{32}{25}$
\solution $\cos^{2}{\alpha} + \sin^{2}{\beta} = \frac{9}{25} + \frac{9}{25} = \frac{18}{25}$
\end{solutionswithpreamble}

\begin{solutions}{Page 30}
\solution %1
$\sin^{2}{\alpha} + \cos^{2}{\alpha} = \left(\dfrac{4}{5}\right)^2 + \left(\dfrac{3}{5}\right)^2 = \dfrac{16}{25} + \dfrac{9}{25} = \dfrac{25}{25} = 1$

\solution %2
It's not an error. According to the corollary of the Pythagoream Theorem, this a right triangle: $a^2 + b^2 = 3^2 + 4^2 = 9 + 16 = 25 = c^2$.

\solution %3
$\sin^{2}{\beta} + \cos^{2}{\beta} = \left(\dfrac{3}{5}\right)^2 + \left(\dfrac{4}{5}\right)^2 = \dfrac{9}{25} + \dfrac{16}{25} = \dfrac{25}{25} = 1$

\solution %4
$\begin{aligned}[t]
\cos^2{\alpha} + \sin^{2}{\alpha} &= 1 \\
\cos^2{\alpha} &= 1 - \sin^{2}{\alpha} = 1- \left(\frac{5}{13}\right)^2 = 1 - \frac{25}{169} = \frac{144}{169} \\
\cos{\alpha} &= \sqrt{\frac{144}{169}} = \frac{12}{13} 
\end{aligned}$

\solution %5
$\begin{aligned}[t]
\cos^2{\alpha} + \sin^{2}{\alpha} &= 1 \\
\cos^2{\alpha} &= 1 - \sin^{2}{\alpha} = 1- \left(\frac{5}{7}\right)^2 = 1 - \frac{25}{49} = \frac{24}{49} \\
\cos{\alpha} &= \sqrt{\frac{24}{49}} = \frac{\sqrt{4}\sqrt{6}}{\sqrt{49}} = \frac{2\sqrt{6}}{7}
\end{aligned}$

\solution %6
We will follow the proof at the bottom of Page 29:
\begin{align*}
sin^{2}{\alpha} + \sin^{2}{\beta} &= \left(\frac{a}{c}\right)^2 + \left(\frac{b}{c}\right)^2 \\
&= \frac{a^2}{c^2} + \frac{b^2}{c^2} \\
&= \frac{a^2 + b^2}{c^2} \\
&= \frac{a^2 + b^2}{a^2 + b^2} \\
&= 1
\end{align*}

\solution %7
Again, we will follow the proof at the bottom of Page 29:
\begin{align*}
cos^{2}{\alpha} + \cos^{2}{\beta} &= \left(\frac{b}{c}\right)^2 + \left(\frac{a}{c}\right)^2 \\
&= \frac{b^2}{c^2} + \frac{a^2}{c^2} \\
&= \frac{a^2 + b^2}{c^2} \\
&= \frac{a^2 + b^2}{a^2 + b^2} \\
&= 1
\end{align*}
\end{solutions}

\begin{solutions}{Page 31}
\solution ~ %1
\begin{center}
\bgroup
\def\arraystretch{2}
\setlength\tabcolsep{15pt}
\begin{tabular}{ |c|c|c| }
\hline
angle $x$ & $\sin{x}$ & $\cos{x}$ \\
\hline
$30\degree$ & $\frac{1}{2}$        & $\frac{\sqrt{3}}{2}$ \\
$45\degree$ & $\frac{1}{\sqrt{2}}$ & $\frac{1}{\sqrt{2}}$ \\ 
$60\degree$ & $\frac{\sqrt{3}}{2}$ & $\frac{1}{2}$        \\
$\alpha$    & $\frac{4}{5}$        & $\frac{3}{5}$        \\
$\beta$     & $\frac{3}{5}$        & $\frac{4}{5}$        \\
\hline
\end{tabular}
\egroup
\end{center}

\solution %2
$\cos{30\degree} = \frac{\sqrt{3}}{2} = \sin{60\degree}$

\solution %3
$\sin^{2}{30\degree} + \cos^{2}{30\degree} = \left(\dfrac{1}{2}\right)^{2} + \left(\dfrac{\sqrt{3}}{2}\right)^{2} = \dfrac{1}{4} + \dfrac{3}{4} = 1$

\solution %4
We can observe from the table that $\sin{x}$ increases with the size of an acute angle ($\sin{30\degree} < \sin{45\degree} < \sin{60\degree}$), while $\cos{x}$ decreases with the size of an acute angle. You can compare the fractions or convert to decimal make sure. We know that $\sin{\alpha} = \frac{4}{5}$. We also know that $\alpha$ is an acute angle.\\
\textit{Is it larger or smaller than $30\degree$?} Larger, $\frac{4}{5} > \frac{1}{2}$ so $\sin{\alpha} > \sin{30\degree}$.\\
\textit{Than $45\degree$?} Larger, $\frac{4}{5} > \frac{1}{\sqrt{2}}$ so $\sin{\alpha} > \sin{45\degree}$.\\
\textit{Than $60\degree$?} Smaller, $\frac{4}{5} < \frac{\sqrt{3}}{2}$ so $\sin{\alpha} < \sin{60\degree}$.

\end{solutions}

\begin{solutions}{Page 33 (First)}
\solution %1
As the angle $\alpha$ get smaller, the ratio of the opposite side to the hypotenuse approaches 0.

\solution %2
Recall from the theorem on page 28 that if $\alpha + \beta = 90\degree$, then $\sin{\alpha} = \cos{\beta}$ and $\cos{\alpha} = \sin{\beta}$. So, if $\sin{90\degree} = 1$, then $\cos{0\degree} = 1$.

\solution %3
$\sin^{2}{0\degree} + \cos^{2}{0\degree} = 0^2 + 1^2 = 0 + 1 = 1$

\solution %4
$\sin^{2}{90\degree} + \cos^{2}{90\degree} = 1^2 + 0^2 = 1 + 0 = 1$

\solution %5
Our friend is mistaken; the sine of an angle can never be greater than 1.
\end{solutions}

\begin{solutions}{Page 33 (Second)}
\solution ~%1
\begin{center}
\bgroup
\def\arraystretch{2}
\setlength\tabcolsep{15pt}
\begin{tabular}{ |c|c|c| }
\hline
$\sin{0\degree} + \cos{0\degree}$ & $0 + 1$        & $1$ \\
\hline
$\sin{30\degree} + \cos{30\degree}$ & $\frac{1}{2} + \frac{\sqrt{3}}{2}$        & $1.366$ (approx.) \\
\hline
$\sin{45\degree} + \cos{45\degree}$ & $\frac{1}{\sqrt{2}} + \frac{1}{\sqrt{2}}$ & $1.414$ (approx.) \\
\hline
$\sin{60\degree} + \cos{60\degree}$ & $\frac{\sqrt{3}}{2} + \frac{1}{2}$        & $1.366$ (approx.) \\
\hline
\makecell{$\sin{\alpha} + \cos{\alpha}$, where $\alpha$\\ is the smaller\ldots} & $\frac{3}{5} + \frac{4}{5}$ & $1.4$ \\
\hline
\makecell{$\sin{\alpha} + \cos{\alpha}$, where $\alpha$\\ is the larger\ldots}  & $ \frac{4}{5} + \frac{3}{5}$ & $1.4$ \\
\hline
\end{tabular}
\egroup
\end{center}

\solution %2
If $\sin{\alpha} = 1$, then $\cos{\alpha} = 0$ and $\sin{\alpha} + \cos{\alpha} = 1$.
If $\cos{\alpha} = 1$, then $\sin{\alpha} = 0$ and $\sin{\alpha} + \cos{\alpha} = 1$.
Otherwise, $\sin{\alpha} < 1$ and $\cos{\alpha} < 1$, so $\sin{\alpha} + \cos{\alpha} < 2$.

\solution %3
First we will expand and simplify $(\sin{\alpha} + \cos{\alpha})^2$:
\begin{align*}
(\sin{\alpha} + \cos{\alpha})^2 &= \sin^{2}{\alpha} + 2\sin{\alpha}\cos{\alpha} + \cos^{2}{\alpha} \\
&= (\sin^{2}{\alpha} + \cos^{2}{\alpha}) + 2\sin{\alpha}\cos{\alpha} \\
&= 1 + 2\sin{\alpha}\cos{\alpha}
\end{align*}
We know that $0 \leq \sin{\alpha} \leq 1$ and $0 \leq \cos{\alpha} \leq 1$ because $\alpha$ is acute.
So $2\sin{\alpha}\cos{\alpha}$ is the product of three nonnegative numbers, and is itself a nonnegative number.
A nonnegative number added to 1 results in a number $\geq 1$.
Therefore, $1 + 2\sin{\alpha}\cos{\alpha} \geq 1$.
The square root of a number $\geq 1$ is itself $\geq 1$.
Therefore, $\sqrt{1 + 2\sin{\alpha}\cos{\alpha}} \geq 1$.
Rewriting the expression on the left: $\sqrt{1 + 2\sin{\alpha}\cos{\alpha}} = \sqrt{(\sin{\alpha} + \cos{\alpha})^2} = \sin{\alpha} + \cos{\alpha}$.
So, $\sin{\alpha} + \cos{\alpha} \geq 1$.

\solution %4
$\sin{45\degree} + \cos{45\degree} = \dfrac{1}{\sqrt{2}} + \dfrac{1}{\sqrt{2}} = \dfrac{2}{\sqrt{2}} = \dfrac{2\sqrt{2}}{\sqrt{2}\sqrt{2}} = \dfrac{2\sqrt{2}}{2} = \sqrt{2}$

\solution %5
You should notice that the values for $\sin{\alpha} + \cos{\alpha}$ increases with larger $alpha$ when $0\degree \leq \alpha < 45\degree$, reaches a maximum value when $\alpha = 45\degree$, then decreases with larger $\alpha$ when $45\degree < \alpha \leq 90\degree$.

\end{solutions}

\begin{solutions}{Page 35}
\solution ~
\begin{center}
\bgroup
\def\arraystretch{2}
\setlength\tabcolsep{15pt}
\begin{tabular}{ |c|c|c| }
\hline
$(\sin{0\degree})(\cos{0\degree})$   & $0\cdot1$        & $0$ \\
\hline
$(\sin{30\degree})(\cos{30\degree})$ & $\frac{1}{2}\cdot\frac{\sqrt{3}}{2}$ & $0.433$ (approx.) \\
\hline
$(\sin{45\degree})(\cos{45\degree})$ & $\frac{1}{\sqrt{2}}\cdot\frac{1}{\sqrt{2}}$ & $0.5$ \\
\hline
$(\sin{60\degree})(\cos{60\degree})$ & $\frac{\sqrt{3}}{2}\cdot\frac{1}{2}$ & $0.433$ (approx.) \\
\hline
\makecell{$(\sin{\alpha})(\cos{\alpha})$, where $\alpha$\\ is the smaller\ldots}  & $\frac{3}{5}\cdot\frac{4}{5}$ & $0.48$ \\
\hline
\makecell{$(\sin{\alpha})(\cos{\alpha})$, where $\alpha$\\ is the larger\ldots}  & $\frac{4}{5}\cdot\frac{3}{5}$ & $0.48$ \\
\hline
\end{tabular}
\egroup
\end{center}
\textit{How large can the product $(\sin{\alpha})(\cos{\alpha})$ get?}
We can see from the table that the maximum value of the product appears to be when $\alpha = 45\degree$.
\end{solutions}

\begin{solutions}{Page 37}
\solution %1
$\cos{\alpha}=3/5$, $\cos{\beta}=4/5$, $\sin{\alpha}=4/5$, $\sin{\beta}=3/5$, $\tan{\alpha}=4/3$, $\tan{\beta}=3/4$, $\cot{\alpha}=3/4$, $\cot{\beta}=4/3$.

\solution %2
We can show that this assumption is correct using the corollary of the Pythagorean Theorem:
$a^2 + b^{2} = 3^{2} + 4^{2} = 25 = c^2$.

\solution %3.
$\cos{\alpha}=a/c$, $\cos{\beta}=b/c$, $\sin{\alpha}=b/c$, $\sin{\beta}=a/c$, $\tan{\alpha}=b/a$, $\tan{\beta}=a/b$, $\cot{\alpha}=a/b$, $\cot{\beta}=b/a$.

\solution %4
$c = \sqrt{12^2 + 5^2} = \sqrt{169} = 13$. $\cos{\alpha}=12/13$. $\cos{\beta}=5/13$. $\cot{\alpha}=12/5$. $\cot{\beta}=5/12$.

\solution %5
First, we will use the Pythagorean Theorem to find the length of the longer leg:
\begin{align*}
a^2 + b^2 &= c^2\\
a^2 + 7^2 &= 25^2\\
a^2 + 49 &= 625\\
a^2 &= 576\\
a &= 24
\end{align*}
We can now find the numerical values that were asked for: $\cos{\alpha}=24/25$, $\cos{\beta}=7/25$, $\cot{\alpha}=24/7$, $\cot{\beta}=7/24$.

\solution %6
$\begin{aligned}[t]
\frac{a}{c} &= \sin{\alpha} = \cos{\beta}\\
\frac{b}{c} &= \cos{\alpha} = \sin{\beta}\\
\frac{a}{b} &= \tan{\alpha} = \cot{\beta}\\
\frac{b}{a} &= \cot{\alpha} = \tan{\beta}
\end{aligned}$

\solution %7
First, we will use the Pythagorean Theorem to find the length of the other leg:
\begin{align*}
a^2 + b^2 &= c^2\\
a^2 + 3^2 &= 5^2\\
a^2 + 9 &= 25\\
a^2 &= 16\\
a &= 4
\end{align*}
We can now find the numerical values that were asked for: $\cos{\alpha}=4/5$, $\cot{\alpha}=4/3$.

\solution %8
If $\tan{\alpha} = 1$, then $a/b=1$, implying that $a = b$ and $\alpha = 45\degree$. $\cos{\alpha} = \cos{45\degree} = 1/\sqrt{2}$. $\cot{\alpha} = 1/1 = 1$. 

\solution %9
$\tan{45\degree} = 1/1 = 1$.

\solution %10
$\tan{30\degree} = 1/\sqrt{3} \approx 0.57735$.

\solution %11
$\tan{45\degree} + \sin{30\degree} = 1 + \frac{1}{2} = \frac{3}{2}$. We don't need a calculator because both numbers are rational.

\end{solutions}


\section*{Chapter 2: Relations among Trigonometric Ratios}

\begin{solutions}{Page 43}
\solution %1
$\begin{aligned}[t]
\cos{\alpha} &= \sqrt{1 - \left(\frac{8}{17}\right)^2} = \sqrt{1 - \frac{64}{289}} = \sqrt{\frac{225}{289}} = \frac{15}{17} \\
\tan{\alpha} &= \frac{\frac{8}{17}}{\frac{15}{17}} = \frac{8}{15} \\
\cot{\alpha} &= \frac{15}{8}
\end{aligned}$

\solution %2
Let the length of the adjacent leg $a$ be $\frac{3}{7}$ and the length of the hypotenuse be 1 (see the first triangle diagram on page 44).
\begin{align*}
\sin{\alpha} &= \sqrt{1 - a^2} = \sqrt{1 - \left(\frac{3}{7}\right)^2} = \sqrt{1 - \frac{9}{49}} = \sqrt{\frac{40}{49}} &= \frac{\sqrt{4}\sqrt{10}}{\sqrt{49}} = \frac{2\sqrt{10}}{7} \\
\tan{\alpha} &= \frac{\sqrt{1 - a^2}}{a} = \frac{\frac{2\sqrt{10}}{7}}{\frac{3}{7}} = \frac{2\sqrt{10}}{3} \\
\cot{\alpha} &= \frac{a}{\sqrt{1 - a^2}} = \frac{3}{2\sqrt{10}}
\end{align*}

\solution %3
$\sin{\alpha} = \sqrt{1 - b^2},\;
\tan{\alpha} = \dfrac{\sqrt{1 - b^2}}{b},\; 
\cot{\alpha} = \dfrac{b}{\sqrt{1 - b^2}}$

\solution %4
$\sin{\alpha} = \dfrac{d}{\sqrt{1 + d^2}},\;
\tan{\alpha} = \dfrac{1}{\sqrt{1 + d^2}},\; 
\cot{\alpha} = \dfrac{1}{d}$

\solution ~ %5
\begin{center}
\bgroup
\def\arraystretch{2.1}
\setlength\tabcolsep{15pt}
\begin{tabular}{ |c|c|c|c|c| }
\hline
~              & $\sin{\alpha}$             & $\cos{\alpha}$             & $\tan{\alpha}$             & $\cot{\alpha}$ \\
\hline
$\sin{\alpha}$ & $a$                        & $\sqrt{1 - a^2}$           & $\dfrac{a}{\sqrt{1 - a^2}}$ & $\dfrac{\sqrt{1 - a^2}}{a}$ \\
\hline
$\cos{\alpha}$ & $\sqrt{1 - a^2}$           & $a$                        & $\dfrac{\sqrt{1 - a^2}}{a}$ & $\dfrac{a}{\sqrt{1 - a^2}}$ \\
\hline
$\tan{\alpha}$ & $\dfrac{a}{\sqrt{1 + a^2}}$ & $\dfrac{1}{\sqrt{1 + a^2}}$ & $a$                       & $\dfrac{1}{a}$ \\
\hline
$\cot{\alpha}$ & $\dfrac{1}{\sqrt{1 + a^2}}$ & $\dfrac{a}{\sqrt{1 + a^2}}$ & $\dfrac{1}{a}$              & $a$ \\
\hline
\end{tabular}
\egroup
\end{center}


\end{solutions}

\begin{solutions}{Page 45 (First)}
\solution %1
Given in text

\solution %2
$\sin^{2}{45\degree} = \left(\dfrac{1}{\sqrt{2}}\right)^2 = \dfrac{1}{2}$

\solution ~ %3
\begin{center}
\bgroup
\def\arraystretch{2.1}
\setlength\tabcolsep{15pt}
\begin{tabular}{ |c|c|c|c|c| }
\hline
~              & $\sin{\alpha}$                                      & $\cos{\alpha}$                           & $\tan{\alpha}$                                      & $\cot{\alpha}$ \\
\hline
$\sin{\alpha}$ & $\sin{\alpha}$                                      & $\sqrt{1 - \sin^{2}{\alpha}}$            & $\dfrac{a}{\sqrt{1 - \sin^{2}{\alpha}}}$            & $\dfrac{\sqrt{1 - \sin^{2}{\alpha}}}{\sin{\alpha}}$ \\
\hline
$\cos{\alpha}$ & $\sqrt{1 - \cos^{2}{\alpha}}$                       & $\cos{\alpha}$                           & $\dfrac{\sqrt{1 - \cos^{2}{\alpha}}}{\cos{\alpha}}$ & $\dfrac{\cos{\alpha}}{\sqrt{1 - \cos^{2}{\alpha}}}$ \\
\hline
$\tan{\alpha}$ & $\dfrac{\tan{\alpha}}{\sqrt{1 + \tan^{2}{\alpha}}}$ & $\dfrac{1}{\sqrt{1 + \tan^{2}{\alpha}}}$ & $\tan{\alpha}$                                      & $\dfrac{1}{\tan{\alpha}}$ \\
\hline
$\cot{\alpha}$ & $\dfrac{1}{\sqrt{1 + \cot^{2}{\alpha}}}$            & $\dfrac{a}{\sqrt{1 + \cot^{2}{\alpha}}}$ & $\dfrac{1}{\cot{\alpha}}$                           & $\cot{\alpha}$ \\
\hline
\end{tabular}
\egroup
\end{center}
\end{solutions}


\begin{solutions}{Page 45 (Second)}
\solution $\tan{\alpha} = \dfrac{a}{b} = \cot{\beta}$
\solution $\cot{\alpha} = \dfrac{b}{a} = \tan{\beta}$
\solution $\sec{\alpha} = \dfrac{c}{a} = \csc{\beta}$
\solution $\csc{\alpha} = \dfrac{c}{b} = \sec{\beta}$
\end{solutions}


\begin{solutions}{Page 47}
\solution %1
\begin{subsolutions}
\subsolution $\sin^{2}{30\degree} + \cos^{2}{30\degree} = \left(\dfrac{1}{2}\right)^2 + \left(\dfrac{\sqrt{3}}{2}\right)^2 = \dfrac{1}{4} + \dfrac{3}{4} = 1$
\subsolution $\sin^{2}{45\degree} + \cos^{2}{45\degree} = \left(\dfrac{1}{\sqrt{2}}\right)^2 + \left(\dfrac{1}{\sqrt{2}}\right)^2 = \dfrac{1}{2} + \dfrac{1}{2} = 1$
\subsolution $\sin^{2}{60\degree} + \cos^{2}{60\degree} = \left(\dfrac{\sqrt{3}}{2}\right)^2 + \left(\dfrac{1}{2}\right)^2= \dfrac{3}{4} + \dfrac{1}{4} = 1$
\end{subsolutions}

\solution %2
$\begin{aligned}[t]
\sin^{2}{\alpha} + \cos^{2}{\alpha} &= 1 \\
\left(\dfrac{\sqrt{5}}{4}\right)^2 + \cos^{2}{\alpha} &= 1 \\
\cos^{2}{\alpha} &= 1 - \left(\frac{\sqrt{5}}{4}\right)^2 = 1 - \frac{5}{16} = \frac{11}{16} \\
\cos{\alpha} &= \sqrt{\frac{11}{16}} = \frac{\sqrt{11}}{4}
\end{aligned}$

\solution %3
$\begin{aligned}[t]
\sin^{2}{\alpha} + \cos^{2}{\alpha} &= 1 \\
\sin^{2}{\alpha} + \left(\frac{2}{3}\right)^2 &= 1 \\
\sin^{2}{\alpha} &= 1 - \frac{4}{9} = \frac{5}{9} \\
\sin{\alpha} &= \sqrt{\frac{5}{9}} = \frac{\sqrt{5}}{3}
\end{aligned}$

\solution %4
\begin{align*}
\frac{\sin{\alpha}}{\cos{\alpha}} &= \tan{\alpha} = \frac{1}{\sqrt{3}} \\
\frac{\sin^{2}{\alpha}}{\cos^{2}{\alpha}} &= \frac{1}{3} \\
\frac{\sin^{2}{\alpha}}{1 - \sin^{2}{\alpha}} &= \frac{1}{3} \\
3\sin^{2}{\alpha} &= 1 - \sin^{2}{\alpha} \\
4\sin^{2}{\alpha} &= 1 \\
\sin^{2}{\alpha} &= \frac{1}{4} \\
\sin{\alpha} &= \sqrt{\frac{1}{4}} = \frac{1}{2}
\end{align*}

\solution %5
\begin{subsolutions}
\subsolution %a
$\cot{x}\sin{x} = \left(\dfrac{1}{\tan{x}}\right)\sin{x} = \dfrac{\sin{x}}{\tan{x}} = \dfrac{\sin{x}}{\frac{\sin{x}}{\cos{x}}} = \dfrac{\sin{x}\cos{x}}{\sin{x}} = \cos{x}$


\subsolution %b
$\dfrac{\tan{x}}{\sin{x}} = \dfrac{\frac{\sin{x}}{\cos{x}}}{\sin{x}} = \dfrac{\frac{\sin{x}}{\cos{x}}\cdot\frac{1}{\sin{x}}}{\sin{x}\cdot\frac{1}{\sin{x}}} = \dfrac{\frac{\sin{x}}{\sin{x}\cos{x}}}{1} = \dfrac{\sin{x}}{\sin{x}\cos{x}} = \dfrac{1}{\cos{x}}$


\subsolution %c
$\cos^{2}{\alpha} - \sin^{2}{\alpha} = \cos^{2}{\alpha} - (1 - \cos^{2}{\alpha}) = \cos^{2}{\alpha} - 1 + \cos^{2}{\alpha} = 2\cos^{2}{\alpha} - 1$

\subsolution %d
This one is tricky. You might need to try a few different approaches (squaring above and below, multiplying above and below by $\cos{\alpha}\sin{\alpha}$). Eventually it becomes clear that you need to multiply above and below by $(1 - \cos{\alpha})$ and find a way to cancel out the $\sin{\alpha}$ factor in the numerator:
\begin{align*}
\frac{\sin{\alpha}}{1 + \cos{\alpha}} &= \frac{\sin{\alpha}(1 - \cos{\alpha})}{(1 + \cos{\alpha})(1 - \cos{\alpha})} = \frac{\sin{\alpha}(1 - \cos{\alpha})}{1 - \cos{\alpha} + \cos{\alpha} - \cos^{2}{\alpha}} \\
&= \frac{\sin{\alpha}(1 - \cos{\alpha})}{1 - \cos^{2}{\alpha}} = \frac{\sin{\alpha}(1 - \cos{\alpha})}{1 - (1 - \sin^{2}{\alpha})} = \frac{\sin{\alpha}(1 - \cos{\alpha})}{1 - 1 + \sin^{2}{\alpha}} \\
&= \frac{\sin{\alpha}(1 - \cos{\alpha})}{\sin^{2}{\alpha}} = \frac{1 - \cos{\alpha}}{\sin{\alpha}}
\end{align*}

\subsolution %e
\begin{align*}
\frac{\sin^{2}{\alpha} + 2\cos^{2}{\alpha} - 1}{\cot^{2}{\alpha}} &=  \frac{1 - \cos^{2}{\alpha} + 2\cos^{2}{\alpha} - 1}{\cot^{2}{\alpha}} = \frac{\cos^{2}{\alpha}}{\left(\frac{\cos{\alpha}}{\sin{\alpha}}\right)^2} \\
&=  \frac{\cos^{2}{\alpha}}{\frac{\cos^{2}{\alpha}}{\sin^{2}{\alpha}}} = \frac{\cos^{2}{\alpha}\sin^{2}{\alpha}}{\cos^{2}{\alpha}} \\
&= \sin^{2}{\alpha}
\end{align*}

\subsolution %f
\begin{align*}
\cos^{2}{\alpha} &= \frac{\cos^{2}{\alpha}}{1} = \frac{\cos^{2}{\alpha}}{\cos^{2}{\alpha} + \sin^{2}{\alpha}} \\
&=  \frac{\frac{\cos^{2}{\alpha}}{\cos^{2}{\alpha}}}{\frac{\cos^{2}{\alpha} + \sin^{\alpha}}{\cos^{2}{\alpha}}} = \frac{1}{\frac{\cos^{2}{\alpha}}{\cos^{2}{\alpha}} + \frac{\sin^{2}{\alpha}}{\cos^{2}{\alpha}}} \\
&= \frac{1}{1 + \tan^{2}{\alpha}}
\end{align*}


\subsolution %g
\begin{align*}
\sin^{2}{\alpha} &= \frac{\sin^{2}{\alpha}}{1} = \frac{\cos^{2}{\alpha}}{\cos^{2}{\alpha} + \sin^{2}{\alpha}} \\
&=  \frac{\frac{\sin^{2}{\alpha}}{\sin^{2}{\alpha}}}{\frac{\cos^{2}{\alpha} + \sin^{\alpha}}{\sin^{2}{\alpha}}} = \frac{1}{\frac{\cos^{2}{\alpha}}{\sin^{2}{\alpha}} + \frac{\sin^{2}{\alpha}}{\sin^{2}{\alpha}}} \\
&= \frac{1}{\cot^{2}{\alpha} + 1}
\end{align*}

\subsolution %h
\begin{align*}
\frac{1 - \cos{\alpha}}{1 + \cos{\alpha}} &= \frac{(1 - \cos{\alpha})(1 + \cos{\alpha})}{(1 + \cos{\alpha})(1 + \cos{\alpha})} = \frac{1 + \cos{\alpha} - \cos{\alpha} - \cos^{2}{\alpha}}{(1 + \cos{\alpha})^2} \\
&= \frac{1 - \cos^{2}{\alpha}}{(1 + \cos{\alpha})^2} = \frac{\sin^{2}{\alpha}}{(1 + \cos{\alpha})^2} \\
&= \left(\frac{\sin{\alpha}}{1 + \cos{\alpha}}\right)^2
\end{align*}

\subsolution %i
The key to solving this one is the formula for factoring a difference of cubes: $a^3 - b^3 = (a - b)(a^2 + ab + b^2)$.
\begin{align*}
\frac{\sin^{3}{\alpha} - \cos^{3}{\alpha}}{\sin{\alpha} - \cos{\alpha}} &= \frac{(\sin{\alpha} - \cos{\alpha})(\sin^{2}{\alpha} + \sin{\alpha}\cos{\alpha} + \cos^{2}{\alpha})}{\sin{\alpha} - \cos{\alpha}} \\
&= \sin^{2}{\alpha} + \sin{\alpha}\cos{\alpha} + \cos^{2}{\alpha} \\
&= 1 + \sin{\alpha}\cos{\alpha}
\end{align*}
\end{subsolutions}

\solution %6
\begin{subsolutions}

\subsolution %a
We can rewrite the LHS to show that $\sin^{4}{\alpha} - \cos^{4}{\alpha} = \cos^{2}{\alpha} - \sin^{2}{\alpha}$:
\begin{align*}
\sin^{4}{\alpha} - \cos^{4}{\alpha} &= (\sin^{2}{\alpha} + \cos^{2}{\alpha})(\sin^{2}{\alpha} - \cos^{2}{\alpha}) = 1(\sin^{2}{\alpha} - \cos^{2}{\alpha}) \\
&= \sin^{2}{\alpha} - \cos^{2}{\alpha}
\end{align*}
Answer: There are no angles $\alpha$ for which $\sin^{4}{\alpha} - \cos^{4}{\alpha} > \cos^{2}{\alpha} - \sin^{2}{\alpha}$ because the expressions on either side of the inequality are equivalent.

\subsolution %b
$\sin^{4}{\alpha} - \cos^{4}{\alpha} >= \cos^{2}{\alpha} - \sin^{2}{\alpha}$ for all angles $\alpha$ because the expressions on either side of the inequality are equivalent.

\end{subsolutions}

\solution %7
If we rewrite $2\sin{\alpha}\cos{\alpha}$ as a fraction, we can divide above and below by $\cos{\alpha}$ to convert the numerator and denominator into expressions in terms of $\tan{\alpha}$:
\begin{align*}
2\sin{\alpha}\cos{\alpha} &= \frac{2\sin{\alpha}\cos{\alpha}}{1} = \frac{2\sin{\alpha}\cos{\alpha}}{\sin^{2}{\alpha} + \cos^{2}{\alpha}} \\
&= \frac{\frac{2\sin{\alpha}\cos{\alpha}}{\cos^{2}{\alpha}}}{\frac{\sin^{2}{\alpha} + \cos^{2}{\alpha}}{\cos^{2}{\alpha}}} = \frac{\frac{2\sin{\alpha}}{\cos{\alpha}}}{\frac{\sin^{2}{\alpha}}{\cos^{2}{\alpha}} + \frac{\cos^{2}{\alpha}}{\cos^{2}{\alpha}}} \\
&= \frac{2\tan{\alpha}}{\tan^{2}{\alpha} + 1}
\end{align*}
Now we can plug in the given value for $\tan{\alpha}$ to find the value of $2\sin{\alpha}\cos{\alpha}$ in this instance:
\begin{equation*}
2\sin{\alpha}\cos{\alpha} = \frac{2\tan{\alpha}}{\tan^{2}{\alpha} + 1} = \frac{2(\frac{2}{5})}{(\frac{2}{5})^2 + 1} = \frac{\frac{4}{5}}{\frac{4}{25} + 1} = \frac{\frac{4}{5}}{\frac{4}{25} + \frac{25}{25}} = \frac{\frac{20}{25}}{\frac{29}{25}} = \frac{20}{29}
\end{equation*}


\solution %8
First, we will rewrite the expresssion $\cos^{2}{\alpha} - \sin^{2}{\alpha}$ in terms of $\tan{\alpha}$:
\begin{align*}
\cos^{2}{\alpha} - \sin^{2}{\alpha} &= \frac{\cos^{2}{\alpha} - \sin^{2}{\alpha}}{1} = \frac{\cos^{2}{\alpha} - \sin^{2}{\alpha}}{\cos^{2}{\alpha} + \sin^{2}{\alpha}} = \frac{\frac{\cos^{2}{\alpha} - \sin^{2}{\alpha}}{\cos^{2}{\alpha}}}{\frac{\cos^{2}{\alpha} + \sin^{2}{\alpha}}{\cos^{2}{\alpha}}} \\
&= \frac{1 - \tan^{2}{\alpha}}{1 + \tan^{2}{\alpha}}
\end{align*}

\begin{subsolutions}

\subsolution
To find the numerical value of $\cos^{2}{\alpha} - \sin^{2}{\alpha}$ when $\tan{\alpha} = \frac{2}{5}$ we can substitute $\frac{2}{5}$ for $\tan{\alpha}$ in the formula above:
\begin{equation*}
\cos^{2}{\alpha} - \sin^{2}{\alpha} = \frac{1 - \tan^{2}{\alpha}}{1 + \tan^{2}{\alpha}} = \frac{1 - \left(\frac{2}{5}\right)^2}{1 + \left(\frac{2}{5}\right)^2} = \frac{1 - \frac{4}{25}}{1 + \frac{4}{25}} = \frac{\frac{21}{25}}{\frac{29}{25}} = \frac{21}{29}
\end{equation*}

\subsolution
Substituting $r$ for $\tan{\alpha}$ in the formula above:
\begin{equation*}
\cos^{2}{\alpha} - \sin^{2}{\alpha} = \frac{1 - r^2}{1 + r^2}
\end{equation*}

\end{subsolutions}

\solution %9
First, we will rewrite the expresssion in terms of $\tan{\alpha}$:

\begin{equation*}
\frac{\sin{\alpha} - 2\cos{\alpha}}{\cos{\alpha} - 3\sin{\alpha}} = \frac{\frac{\sin{\alpha} - 2\cos{\alpha}}{\cos{\alpha}}}{\frac{\cos{\alpha} - 3\sin{\alpha}}{\cos{\alpha}}} = \frac{\frac{\sin{\alpha}}{\cos{\alpha}} - \frac{2\cos{\alpha}}{\cos{\alpha}}}{\frac{\cos{\alpha}}{\cos{\alpha}} - \frac{3\sin{\alpha}}{\cos{\alpha}}} = \frac{\tan{\alpha} - 2}{1 - 3\tan{\alpha}}
\end{equation*}

Next, we substitute  $\frac{2}{5}$ for $\tan{\alpha}$:
\begin{equation*}
\frac{\tan{\alpha} - 2}{1 - 3\tan{\alpha}} = \frac{\frac{2}{5} - 2}{1 - 3\left(\frac{2}{5}\right)} = \frac{\frac{2}{5} - \frac{10}{5}}{\frac{5}{5} - \frac{6}{5}} = \frac{-\frac{8}{5}}{-\frac{1}{5}} = 8
\end{equation*}


\solution %10
First, we will rewrite the expresssion in terms of $\tan{\alpha}$:
\begin{equation*}
\frac{a\sin{\alpha} + b\cos{\alpha}}{c\cos{\alpha} + d\sin{\alpha}} = \frac{\frac{a\sin{\alpha}}{\cos{\alpha}} + \frac{b\cos{\alpha}}{\cos{\alpha}}}{\frac{c\cos{\alpha}}{\cos{\alpha}} + \frac{d\cos{\alpha}}{\cos{\alpha}}} = \frac{a\tan{\alpha} + b}{c + d\tan{\alpha}}
\end{equation*}

Next, we substitute  $\frac{2}{5}$ for $\tan{\alpha}$ and simplify:
\begin{equation*}
\frac{a\tan{\alpha} + b}{c + d\tan{\alpha}} = \frac{a\left(\frac{2}{5}\right) + b\left(\frac{5}{5}\right)}{c\left(\frac{5}{5}\right) + d\left(\frac{2}{5}\right)} = \frac{\frac{2a + 5b}{5}}{\frac{5c+2d}{5}} = \frac{2a + 5b}{5c + 2d}
\end{equation*}

Now we can see why the problem included the restriction that $5c + 2d \neq 0$; the value of the expression is undefined if the denominator is zero. The sum of two rational numbers is a rational number. Therefore the numerator and denominator in the expression are both rational numbers. The quotient of two rational numbers is a rational number. Therefore, the entire expression evaluates to a rational number for arbitratrary rational values of $a$, $b$, $c$ and $d$.

\solution %11
We can expand and simplify the expression:
\begin{align*}
& (\sin{\alpha} + \cos{\alpha})^2 + (\sin{\alpha} - \cos{\alpha})^2 \\
&= \sin^{2}{\alpha} + 2\sin{\alpha}\cos{\alpha} + \cos^{2}{\alpha} + \sin^{2}{\alpha} - 2\sin{\alpha}\cos{\alpha} + \cos^{2}{\alpha} \\
&= 2\sin^{2}{\alpha} + 2\cos^{2}{\alpha} \\
&= 2(\sin^{2}{\alpha} + \cos^{2}{\alpha}) \\
&= 2(1) \\
&= 2
\end{align*}
As the expression evaluates to a constant, it is as large as possible for all values of $\alpha$.

\end{solutions}

\begin{solutions}{Page 49}

\solution %1

Rewriting any instances of $\sec{\alpha}$ or $\csc{\alpha}$ on either side of the identities:

\begin{subsolutions}

\subsolution %a
$\begin{aligned}[t]
\tan{\alpha}\csc{\alpha} &= \sec{\alpha} \\
\tan{\alpha}\dfrac{1}{\sin{\alpha}} &= \dfrac{1}{\cos{\alpha}} \\[1ex]
\dfrac{\tan{\alpha}}{\sin{\alpha}} &= \dfrac{1}{\cos{\alpha}}
\end{aligned}$

\subsolution %b
$\begin{aligned}[t]
\cot{\alpha}\csc{\alpha} &= \sec{\alpha} \\
\cot{\alpha}\dfrac{1}{\cos{\alpha}} &= \dfrac{1}{\sin{\alpha}} \\[1ex]
\dfrac{\cot{\alpha}}{\cos{\alpha}} &= \dfrac{1}{\sin{\alpha}}
\end{aligned}$

\subsolution %c
$\begin{aligned}[t]
\dfrac{1}{\sec{\alpha}}\csc{\alpha} &= \cot{\alpha} \\
\dfrac{1}{\frac{1}{\cos{\alpha}}}\cdot\dfrac{1}{\sin{\alpha}} &= \cot{\alpha} \\
\cos{\alpha}\dfrac{1}{\sin{\alpha}} &= \cot{\alpha} \\[1ex]
\dfrac{\cos{\alpha}}{\sin{\alpha}} &= \cot{\alpha}
\end{aligned}$

\subsolution %d
$\begin{aligned}[t]
\tan^{2}{\alpha} &= (\sec{\alpha} + 1)(\sec{\alpha} - 1) \\
\tan^{2}{\alpha} &= \sec^{2}{\alpha} - 1 \\
\tan^{2}{\alpha} &= \dfrac{1}{\cos^{2}{\alpha}} - 1
\end{aligned}$

\subsolution %e
$\begin{aligned}[t]
\csc^{2}{\alpha} &= 1 + \cot^{2}{\alpha} \\
\dfrac{1}{\sin^{2}{\alpha}} &= 1 + \cot^{2}{\alpha}
\end{aligned}$

\end{subsolutions}


\solution %2
Rewriting any instances of $\sin{\alpha}$ or $\cos{\alpha}$ on either side of the identities, and eliminating fractions:
% TODO are these correct? Some of the solutions look simplistic
\begin{subsolutions}

\subsolution %a
$\begin{aligned}[t]
\frac{\tan{\alpha}}{\sin{\alpha}} = \frac{1}{\cos{\alpha}} \\
\tan{\alpha}\frac{1}{\sin{\alpha}} = \sec{\alpha} \\
\tan{\alpha}\csc{\alpha} = \sec{\alpha}
\end{aligned}$

\subsolution %b
$\begin{aligned}[t]
\frac{1}{\sin{\alpha}}\cos{\alpha} = \cot{\alpha} \\
\frac{\cos{\alpha}}{\sin{\alpha}} = \cot{\alpha} \\
\cot{\alpha} = \cot{\alpha}
\end{aligned}$

\subsolution %c
$\begin{aligned}[t]
\tan^{2}{\alpha} + 1 = \frac{1}{\cos^{2}{\alpha}} \\
\tan^{2}{\alpha} + 1 = \sec^{2}{\alpha}
\end{aligned}$

\subsolution %d
$\begin{aligned}[t]
\frac{1}{\sin^{2}{\alpha}} = 1 + \cot^{2}{\alpha} \\
\csc^{2}{\alpha} = 1 + \cot^{2}{\alpha}
\end{aligned}$

\end{subsolutions}

\end{solutions}



\begin{solutions}{Page 50}

\solution % 1
First, we find the value of $a^2 + b^2$:
\begin{align*}
a^2 + b^2 &= (\cos^{2}{\alpha} - \sin^{2}{\alpha})^2 + (2\sin{\alpha}\cos{\alpha})^2 \\
&= \cos^{4}{\alpha} - 2\cos^{2}{\alpha}\sin^{2}{\alpha} + \sin^{4}{\alpha} + 4\sin^{2}{\alpha}\cos^{2}{\alpha} \\
&= \cos^{4}{\alpha} + 2\cos^{2}{\alpha}\sin^{2}{\alpha} + \sin^{4}{\alpha} \\
&= (\cos^{2}{\alpha} + \sin^{2}{\alpha})^2 \\
&= (1)^2 \\
&= 1
\end{align*}
According to the lemma on Page 50, as $a^2 + b^2 = 1$, an angle $\theta$ exists such that $a = \cos{\theta}$ and $b = \sin{\theta}$.

\solution %2
First, we find the value of $a^2 + b^2$:
\begin{align*}
a^2 + b^2 &= \left(\sqrt{\frac{1 + \cos{\alpha}}{2}}\right)^2 + \left(\sqrt{\frac{1 - \cos{\alpha}}{2}}\right)^2 \\
&= \frac{1 + \cos{\alpha}}{2} + \frac{1 - \cos{\alpha}}{2} \\
&= \frac{1 + \cos{\alpha} + 1 - \cos{\alpha}}{2} \\
&= \frac{2}{2} \\
&= 1
\end{align*}

\solution %3
First, we will rewrite $a$ and $b$ to eliminate the cube exponents:
\begin{align*}
a &= 4\cos^{3}{\alpha} - 3\cos{\alpha} \\
&= 4\cos{\alpha}\cos^{2}{\alpha} - 3\cos{\alpha} \\
&= 4\cos{\alpha}(1 - \sin^{2}{\alpha}) - 3\cos{\alpha} \\
&= 4\cos{\alpha} - 4\sin^{2}{\alpha}\cos{\alpha} - 3\cos{\alpha} \\
&= \cos{\alpha} - 4\sin^{2}{\alpha}\cos{\alpha}
\end{align*}
\begin{align*}
b &= 3\sin{\alpha} - 4\sin^{3}{\alpha} \\
&= 3\sin{\alpha} - 4\sin{\alpha}\sin^{2}{\alpha} \\
&= 3\sin{\alpha} - 4\sin{\alpha}(1 - \cos^{2}{\alpha}) \\
&= -\sin{\alpha} + 4\sin{\alpha}\cos^{2}{\alpha}
\end{align*}
Next, we will expand $a^2$ and $b^2$:
\begin{align*}
a^2 &= (\cos{\alpha} - 4\sin^{2}{\alpha}\cos{\alpha})^2 \\
&= \cos^{2}{\alpha} - 8\sin^{2}{\alpha}\cos^{2}{\alpha} + 16\sin^{4}{\alpha}\cos^{2}{\alpha}
\end{align*}
\begin{align*}
b^2 &= (-\sin{\alpha} + 4\sin{\alpha}\cos^{2}{\alpha})^2 \\
&= \sin^{2}{\alpha} - 8\sin^{2}{\alpha}\cos^{2}{\alpha} + 16\sin^{2}{\alpha}\cos^{4}{\alpha}
\end{align*}
Next, we add the expressions for $a^2$ and $b^2$ and simplify to 1:
\begin{align*}
a^2 + b^2 &= \cos^{2}{\alpha} - 8\sin^{2}{\alpha}\cos^{2}{\alpha} + 16\sin^{4}{\alpha}\cos^{2}{\alpha} +\sin^{2}{\alpha} - 8\sin^{2}{\alpha}\cos^{2}{\alpha} + \\
&\quad\quad16\sin^{2}{\alpha}\cos^{4}{\alpha} \\
&= \cos^{2}{\alpha} + \sin^{2}{\alpha} - 16\sin^{2}{\alpha}\cos^{2}{\alpha} + 16\sin^{4}{\alpha}\cos^{2}{\alpha} + 16\sin^{2}{\alpha}\cos^{4}{\alpha} \\
&= \cos^{2}{\alpha} + \sin^{2}{\alpha} + 16\sin^{2}{\alpha}\cos^{2}{\alpha}(-1 + \sin^{2}{\alpha} + \cos^{2}{\alpha}) \\
&= 1 + 16\sin^{2}{\alpha}\cos^{2}{\alpha}(0) \\
&= 1
\end{align*}
According to the lemma on Page 50, as $a^2 + b^2 = 1$, an angle $\theta$ exists such that $a = \cos{\theta}$ and $b = \sin{\theta}$.

\solution %4
First, we find the value of $a^2 + b^2$:
\begin{align*}
a^2 + b^2 &= \left(\frac{1 - t^2}{1 + t^2}\right)^2 + \left(\frac{2t}{1 + t^2}\right)^2 \\[1ex]
&= \frac{(1 - t^2)^2}{(1 + t^2)^2} + \frac{(2t)^2}{(1 + t^2)^2} \\[1ex]
&= \frac{(1 - t^2)^2 + (2t)^2}{(1 + t^2)^2} \\[1ex]
&= \frac{1 - 2t^2 + t^4 + 4t^2}{(1 + t^2)(1 + t^2)} \\[1ex]
&= \frac{(1 + t^2)(1 + t^2)}{(1 + t^2)(1 + t^2)} \\
&= 1
\end{align*}
According to the lemma on Page 50, as $a^2 + b^2 = 1$, an angle $\theta$ exists such that $a = \cos{\theta}$ and $b = \sin{\theta}$.

\solution %5
We expand $(p^2 - q^2)^2 + (2pq)^2$ and use the fact that $p^2 + q^2 = 1$ to simplify to 1:
\begin{align*}
(p^2 - q^2)^2 + (2pq)^2 &= p^{4} - 2p^2q^2 + q^4 + 4p^2q^2 \\
&= p^{4} + 2p^2q^2 + q^4 \\
&= (p^2 + q^2)^2 \\
&= (1)^2 \\
&= 1
\end{align*}
This is similar to Exercise 1 above.

\end{solutions}

\begin{comment}
\begin{solutions}{Page 51}
\solution
\solution
\solution
\solution
\solution
\solution
\end{solutions}

\begin{solutions}{Page 52}
\solution
\solution
\solution
\solution
\end{solutions}

\begin{solutions}{Page 53}
\solution
\solution
\solution
\solution
\solution
\solution
\solution
\solution
\solution
\solution
\solution
\solution
\solution
\solution
\end{solutions}

\begin{solutions}{Page 55}
\solution
\solution
\solution
\solution
\end{solutions}

\begin{solutions}{Page 56}
\solution
\solution
\solution
\solution
\end{solutions}

\begin{solutions}{Page 59}
\solution
\solution
\solution
\solution
\solution
\solution
\solution
\solution
\solution
\solution
\solution
\end{solutions}

\begin{solutions}{Page 62}
\solution
\solution
\end{solutions}

\begin{solutions}{Page 64}
\solution
\solution
\solution
\end{solutions}

\begin{solutions}{Page 65}
\solution
\solution
\solution
\solution
\solution
\end{solutions}


\section*{Chapter 3: Relationships in a Triangle}

\begin{solutions}{Page 68}
\solution
\end{solutions}

\begin{solutions}{Page 70}
\solution
\solution
\solution
\solution
\end{solutions}

\begin{solutions}{Page 71}
\solution
\end{solutions}

\begin{solutions}{Page 73}
\solution
\solution
\solution
\solution
\solution
\solution
\solution
\solution
\end{solutions}

\begin{solutions}{Page 75 (First)}
\solution
\solution
\end{solutions}

\begin{solutions}{Page 75 (Second)}
\solution
\solution
\solution
\solution
\solution
\solution
\solution
\solution
\solution
\solution
\solution
\solution
\solution
\end{solutions}

\begin{solutions}{Page 79}
\solution
\end{solutions}

\begin{solutions}{Page 80}
\solution
\solution
\solution
\solution
\solution
\solution
\solution
\solution
\solution
\solution
\solution
\solution
\solution
\solution
\end{solutions}

\begin{solutions}{Page 83}
\solution
\solution
\solution
\end{solutions}

\begin{solutions}{Page 84}
\solution
\solution
\end{solutions}

\begin{solutions}{Page 85}
\solution
\solution
\end{solutions}

\begin{solutions}{Page 86}
\solution
\end{solutions}

\begin{solutions}{Page 88}
\solution
\solution
\solution
\solution
\solution
\end{solutions}


\section*{Chapter 4: Angles and Rotations}

\begin{solutions}{Page 93}
\solution
\end{solutions}

\begin{solutions}{Page 98}
\solution
\solution
\end{solutions}

\begin{solutions}{Page 100}
\solution
\solution
\solution
\solution
\solution
\end{solutions}

\begin{solutions}{Page 102}
\solution
\solution
\solution
\end{solutions}


\section*{Chapter 5: Radian Measure}

\begin{solutions}{Page 107}
\solution
\solution
\solution
\solution
\solution
\solution
\solution
\solution
\solution
\solution
\solution
\solution
\solution
\solution
\end{solutions}

\begin{solutions}{Page 111}
\solution
\solution
\solution
\solution
\solution
\solution
\solution
\solution
\solution
\solution
\solution
\solution
\solution
\solution
\solution
\solution
\solution
\solution
\solution
\solution
\end{solutions}

\begin{solutions}{Page 114}
\solution
\solution
\solution
\solution
\solution
\solution
\solution
\solution
\end{solutions}

\begin{solutions}{Page 116}
\solution
\end{solutions}

\begin{solutions}{Page 120}
\solution
\solution
\solution
\solution
\solution
\solution
\solution
\solution
\solution
\end{solutions}


\section*{Chapter 6: The Addition Formulas}

\begin{solutions}{Page 123}
\solution
\solution
\solution
\solution
\solution
\end{solutions}

\begin{solutions}{Page 125}
\solution
\solution
\solution
\solution
\solution
\solution
\end{solutions}

\begin{solutions}{Page 129}
\solution
\solution
\end{solutions}

\begin{solutions}{Page 131}
\solution
\solution
\solution
\solution
\solution
\solution
\solution
\solution
\solution
\solution
\solution
\solution
\solution
\solution
\solution
\solution
\solution
\end{solutions}


\section*{Chapter 7: Trigonometric Identities}

\begin{solutions}{Page 141}
\solution
\solution
\solution
\solution
\end{solutions}

\begin{solutions}{Page 142}
\solution
\solution
\solution
\solution
\solution
\solution
\solution
\solution
\solution
\solution
\solution
\solution
\end{solutions}

\begin{solutions}{Page 144}
\solution
\solution
\solution
\solution
\solution
\solution
\solution
\solution
\end{solutions}

\begin{solutions}{Page 147}
\solution
\solution
\solution
\solution
\solution
\solution
\solution
\solution
\solution
\solution
\solution
\solution
\end{solutions}

\begin{solutions}{Page 148}
\solution
\solution
\solution
\solution
\solution
\solution
\solution
\solution
\solution
\end{solutions}

\begin{solutions}{Page 150}
\solution
\solution
\solution
\solution
\solution
\solution
\solution
\end{solutions}

\begin{solutions}{Page 152}
\solution
\solution
\end{solutions}

\begin{solutions}{Page 153}
\solution
\solution
\solution
\solution
\solution
\solution
\solution
\end{solutions}

\begin{solutions}{Page 155}
\solution
\solution
\solution
\solution
\solution
\solution
\solution
\solution
\solution
\solution
\solution
\solution
\end{solutions}

\begin{solutions}{Page 158}
\solution
\solution
\end{solutions}

\begin{solutions}{Page 160}
\solution
\solution
\solution
\end{solutions}

\begin{solutions}{Page 161 (First)}
\solution
\solution
\end{solutions}

\begin{solutions}{Page 161 (Second)}
\solution
\solution
\solution
\end{solutions}

\begin{solutions}{Page 162 (First)}
\solution
\solution
\solution
\end{solutions}

\begin{solutions}{Page 162 (Second)}
\solution
\solution
\solution
\solution
\end{solutions}

\begin{solutions}{Page 163 (First)}
\solution
\solution
\end{solutions}

\begin{solutions}{Page 163 (Second)}
\solution
\solution
\solution
\end{solutions}

\begin{solutions}{Page 164}
\solution
\solution
\end{solutions}

\begin{solutions}{Page 166}
\solution
\solution
\solution
\solution
\end{solutions}

\begin{solutions}{Page 168}
\solution
\solution
\solution
\solution
\end{solutions}

\begin{solutions}{Page 170}
\solution
\solution
\solution
\solution
\solution
\end{solutions}


\section*{Chapter 8: Graphs of Trigonometric Functions}

\begin{solutions}{Page 177}
\solution
\solution
\solution
\solution
\solution
\solution
\solution
\solution
\solution
\solution
\solution
\end{solutions}

\begin{solutions}{Page 179}
\solution
\solution
\solution
\solution
\solution
\solution
\solution
\end{solutions}

\begin{solutions}{Page 181}
\solution
\solution
\solution
\solution
\solution
\solution
\solution
\end{solutions}

\begin{solutions}{Page 183}
\solution
\solution
\solution
\solution
\solution
\end{solutions}

\begin{solutions}{Page 183}
\solution
\solution
\solution
\solution
\solution
\solution
\solution
\solution
\solution
\solution
\solution
\solution
\solution
\solution
\solution
\solution
\solution
\end{solutions}

\begin{solutions}{Page 188}
\solution
\solution
\end{solutions}

\begin{solutions}{Page 189}
\solution
\solution
\end{solutions}

\begin{solutions}{Page 191}
\solution
\solution
\solution
\solution
\solution
\solution
\solution
\solution
\end{solutions}

\begin{solutions}{Page 194}
\solution
\solution
\end{solutions}

\begin{solutions}{Page 196}
\solution
\solution
\solution
\solution
\solution
\solution
\solution
\end{solutions}

\begin{solutions}{Page 197}
\solution
\solution
\solution
\solution
\end{solutions}

\begin{solutions}{Page 199}
\solution
\solution
\solution
\solution
\end{solutions}

\begin{solutions}{Page 200}
\solution
\solution
\solution
\solution
\end{solutions}

\begin{solutions}{Page 203}
\solution
\solution
\solution
\solution
\solution
\end{solutions}

\begin{solutions}{Page 205}
\solution
\solution
\solution
\end{solutions}


\section*{Chapter 9: Inverse Functions and Trigonometric Equations}

\begin{solutions}{Page 213}
\solution
\solution
\solution
\solution
\solution
\solution
\solution
\solution
\solution
\solution
\solution
\solution
\solution
\end{solutions}

\begin{solutions}{Page 220}
\solution
\solution
\solution
\solution
\solution
\end{solutions}

\begin{solutions}{Page 221}
\solution
\solution
\solution
\solution
\solution
\solution
\solution
\solution
\solution
\end{solutions}

\begin{solutions}{Page 225}
\solution
\solution
\solution
\solution
\solution
\solution
\solution
\solution
\solution
\solution
\solution
\solution
\solution
\end{solutions}

\begin{solutions}{Page 227}
\solution
\solution
\solution
\solution
\end{solutions}

\begin{solutions}{Page 229}
\solution
\solution
\solution
\solution
\end{solutions}

\end{comment}

\end{document}
